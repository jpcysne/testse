\chapter{Introdução}
\label{cap:introducao}

Os Processos de negócio são trabalhos executados por empresas para otimizar uma melhor experiência entre seus clientes e o produto final. Ao aplicar o Gerenciamento de Processos de Negócio (BPM – Business Process Management) a empresa se concentra nos processos interfuncionais que agregam valor para seus clientes \cite{CBOK}.

O BPM caracteriza-se como uma nova forma de idealizar as operações de negócio que vão além das estruturas funcionais tradicionais das empresas \cite{CBOK}. Essa idealização contém todo o trabalho efetuado para entregar o produto ou serviço do processo para o cliente, independente de quais áreas ou localizações estejam incluídos \cite{CBOK}. Devido ao dinamismo e a colaboração das várias áreas que estão envolvidas nos processos, que a gestão de processo vem ganhando importância \cite{TechAssistBPM}.

Com a utilização de um BPM, há uma simplificação dos processos, desta forma, novos requisitos para trabalho automatizado, assim como suas atribuições e remanejamento podem ser adicionados aos novos modelos e utilizados com o objetivo de gerar as aplicações necessárias para melhorar o controle e o monitoramento do desempenho \cite{CBOK}. Além do mais, o BPM prevê que os processos continuem melhorando ao longo do tempo e que os mesmos se adaptem a novas modificações \cite{TechAssistBPM}. 

O BPMN (Business Process Model and Notation) foi criado pela OMG (Object Management Group) com o objetivo principal é disponibilizar uma notação que possa ser entendida por todos da área de negocio, desde os analistas de negócios, responsáveis por criar os desenhos dos processos até desenvolvedores técnicos responsáveis por implementar as tecnologias que darão suporte a execução destes processos e finalmente, os responsáveis pelo negocio, que monitoram e administram os processos \cite{OMG}.  

O BPMN ajuda as organizações a entender os próprios processos internos, usando-se de uma notação gráfica que facilitar o entendimento para todos e as organizações usaram essa notação para comunicar os processos criados de modo padrão \cite{OMG}. Com a versão 2.0 do BPMN, houve um aumento da capacidade do BPMN de criar diagramas complexos, onde se utilizar de elementos gráficos para a criação dos diagramas dos processos \cite{TechAssistBPM}. Alem do mais, O BPMN disponibilizar um mapeamento que é utilizado nos BPMS \cite{OMG}.



O BPMS consiste em uma ferramenta que fornece suporte e capacitação na disciplina gerencial do BPM e seu objetivo é a construção de um ambiente operacional integrado entre a tecnologia e o negócio \cite{CBOK}. Onde simulam diversas possibilidades de atuação, que são baseados nos cenários reais, de acordo com a rotina da organização, onde sua performance baseia-se nas informações das bases de testes, que por sua vez, são reflexos das informações repassadas pelas empresas \cite{CBOK}. Alem do mais, as aplicações criadas pelo BPMS apresentam integrações com os sistemas legados das empresas e dão apoio aos processos mapeados e as regras de negócios \cite{CBOK}.

Segundo \cite{CBOK}, os principais usos para o BPMS correspondem a análise de processos, atuando principalmente no tempo, custo, capacidade e qualidade, também são utilizados na modelagem e arquitetura de processos, simulação e gerenciamento de dados, desenho, armazenamento e execução de regras de negócio, assim como interface, geração, execução e medição de processos.

Outros benefícios do BPMS consistem em oferecer suporte, a modelagem, o gerenciamento e controle de acesso a dados, a construção de aplicações por meio de modelos, regras e definições de dados, o monitoramento e medição de desempenho, ocorrendo uma redução das atividades e facilita mudanças rápidas por meio de desenhos e testes interativos, da mesma forma que possibilita uma maior interação do cliente com a organização \cite{CBOK}. Acrescente-se que a perda da informação e as redundâncias dos processos são evitados com o uso do BPMS, onde esses dois motivos são responsáveis pela diminuição da competência da organização \cite{TechAssistBPM}.


Com o uso das ferramentas BPMS houve uma maior automatização dos processos das empresas, onde diminuiu o trabalho manual, mas não eliminou totalmente sua interação, pois existem muitas atividades que precisam ser feitas de forma manual, como exemplo: assinatura de documento e cientificação de um contribuinte. 

Durante o ciclo de vida de um desenvolvimento de software, um dos maiores gastos encontram-se em torno da manutenção do sistema. Pelo fato de que sempre que o sistema for evoluindo fica à mercê de novos problemas, devido a isso, a manutenção, é onde se tem maior atividade. Pelo fato da manutenção ser muitas vezes feita de forma manual, se torna uma atividade repetitiva e difícil, devido a isso, é uma atividade com chances de apresentar falhas.  

Esse fato, aplica-se ao desenvolvimento dos processos automatizados utilizando-se das ferramentas BPMS. Por não existir uma ferramenta de teste automatizado que venha a ser considerado apropriada para a utilização conjunta com uma ferramenta BPMS, geralmente utiliza-se de uma combinação de ferramentas para a realização dos testes \cite{TestCaseBPMN}.  

Devido à dificuldade apresentada no paragrafo anterior, que viabilizou-se a necessidade da criação desse trabalho e o projeto pretende contextualizar as necessidades dos seguintes testes: Teste de Unidade, onde se testa a menor unidade do sistema; Teste de Integração, teste que verifica se oque foi desenvolvido ta de igual com oque foi pedido; Teste de Sistema, ; Teste de Aceitação, é onde se testa junto com o cliente, posteriormente apresentar como funcionam, junto com as suas contribuições para a utilização conjunta com a ferramenta BPMS,   Bizagi. Onde são ferramentas, com objetivo de usar a notação BPMN para a criação de aplicações para as organizações. Sendo o Bizagi uma ferramenta \textit{freeware} onde se usa C Sharp.



%//previa do estudo de caso


\section{Motivação}
\label{sec:motivacao}  
	Durante o desenvolvimento de processos automatizados um dos problemas enfrentados era a realização de testes nos processos operacionais, já que os mesmos eram feitos de forma manual, o tempo necessário para realiza-los nas regras de negócios e sua funcionalidade era elevado.
    
%	A finalidade desse trabalho é expor as importância dos testes automatizados para sistemas BPMS, para poder facilitar o desenvolvimento do mesmo. Predente-se assim, mostrar as vantagens de ter uma ferramenta de teste bem definida e com quais ferramentas pode ser usadas para a criação dos testes automatizados para os sistemas BPMS.


\section{Objetivos}
\label{sec:objetivos}
	
\subsection{Objetivo Geral}
\label{sec:objetivo-geral}
	
    Esse projeto objetiva expor as contribuições, o funcionamento e a importância dos testes automatizados, por meio do uso da ferramenta Bizagi.
	
	%Esse projeto embasa-se em mostrar como funcionam os testes automatizados fazendo uso das ferramentas BPMS, onde será abordado a importância dos testes automatizados para o BPMS e especificando cada tipo dos testes automatizados.

\subsection{Objetivos Específicos}
\label{sec:objetivos-especificos}
	Os objetivos gerais que esse projeto apresenta podem ser executados por meio desses objetivos específicos:   
	\begin{alineas}
    	\item Contextualizar a necessidade dos testes automatizados para o BPMS.
		\item Apresentar como o testes automatizados vão funcionar em um BPMS.
		\item Mostrar as contribuinte de se fazer os testes automatizados no BPMS.
		\item Apresentar a ferramenta usada para os testes automatizados para o BPMS.
	\end{alineas}
    
    
\section{Estrutura do Projeto}
\label{sec:Estrutura-do-Projeto}
Além da introdução, esse projeto abrange outros três capítulos.

O capitulo 2 irá aborda a fundamentação teórica sobre os teste, onde abordará sobre o teste automatizado e os tipos de testes que tem em um teste automatizado que serão utilizados no estudo de caso. Alem disso, o capitulo começará expondo sobre conceitos básicos de teste, depois irá aborda os conceitos de teste automatizado. Por fim, terá uma a discussão dos conceitos básicos que pertencem aos testes automatizados.

No capitulo 3 relata a fundamentação teórica sobre as ferramentas BPMS, onde terá informações gerais sobre a ferramenta BPMS e como funciona o mesmo.

No capitulo 4 será abordado sobre a metodologia que será usado nesse trabalho. Onde também será discutido sobre o projeto BPM usado nesse trabalho, as ferramentas BPMS escolhidas, como serão feito os testes baseado nos tipos de testes discutido no capitulo 2 e quais ferramentas serão utilizadas.

No capitulo 5 será abordado sobre  estudo de caso desse trabalho. Onde será feito abordando as ferramenta de teste selecionadas, junto com a ferramentas BPMS escolhidas(Bizagi e Bonita), que ambas tem o mesmo projeto e modelo de dados e terão os mesmo testes realizados.

No capitulo  6 vai ser dito os resultados do estudo de caso, onde apresentará como os testes foram em ambas as ferramentas BPMS e por fim será mostrado o resultado final em cada ferramenta, discutindo a sua importância.

